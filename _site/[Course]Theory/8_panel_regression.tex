\RequirePackage{currfile}
\tolerance=5000
\documentclass[10pt, xcolor=x11names,compress,usenames,dvipsnames]{beamer}

\usepackage[english]{babel}

\usepackage[framemethod=TikZ]{mdframed}

\usepackage{amssymb,amsmath,amsfonts,eurosym,geometry,ulem,graphicx,caption,color,setspace,comment,footmisc,caption,pdflscape,subfigure,array,hyperref,upgreek,bbm,xcolor,float,amsthm,amsmath,verbatim,setspace,ulem,textpos,changepage,url,multirow,tikz,color, colortbl,numprint,mathrsfs,cancel,wrapfig,booktabs,threeparttable,ebgaramond,natbib}

\usetikzlibrary{fit,shapes.geometric}

\newcounter{nodemarkers}
\newcommand\circletext[1]{%
    \tikz[overlay,remember picture]
        \node (marker-\arabic{nodemarkers}-a) at (0,1.5ex) {};%
    #1%
    \tikz[overlay,remember picture]
        \node (marker-\arabic{nodemarkers}-b) at (0,0){};%
    \tikz[overlay,remember picture,inner sep=2pt]
        \node[draw,rectangle,red ,fit=(marker-\arabic{nodemarkers}-a.center) (marker-\arabic{nodemarkers}-b.center)] {};%
    \stepcounter{nodemarkers}%
}


\setbeamertemplate{footline}[frame number]


\normalem

\newcommand{\tsout}[1]{\text{\sout{$#1$}}}
\definecolor{Gray}{gray}{0.9}
\newcolumntype{g}{>{\columncolor{Gray}}c}

\newtheorem{remark}{Remark}
\def\mb{\mathbf}
\def\iid{\mathrm{i.i.d.}}
\def\bs{\boldsymbol}
\def\tbf{\textbf}
\def\t{^{\top}}
\def\E{\mathbbm{E}}
\def\bSig{\bs{\Sigma}}

\newcommand{\mcitet}[1]{\mbox{\citet{#1}}}
\newcommand{\mcitep}[1]{\mbox{\citep{#1}}}
\newcommand{\ind}{\mathbbm{1}}

\DeclareMathOperator{\vect}{vec}
\DeclareMathOperator{\vecth}{vech}



\newcommand{\R}{\mathbbm{R}}
\newcommand{\prob}{\mathbbm{P}}
\newcommand{\var}{\text{var}}

\newtheorem{assumption}{Assumption}

\newtheorem{hyp}{Hypothesis}
\newtheorem{subhyp}{Hypothesis}[hyp]
\renewcommand{\thesubhyp}{\thehyp\alph{subhyp}}

\newcommand{\red}[1]{{\color{red} #1}}
\newcommand{\blue}[1]{{\color{blue} #1}}

\makeatletter
\newenvironment<>{proofs}[1][\proofname]{%
    \par
    \def\insertproofname{#1\@addpunct{.}}%
    \usebeamertemplate{proof begin}#2}
  {\usebeamertemplate{proof end}}
\makeatother


\newcolumntype{L}[1]{>{\raggedright\let\newline\\arraybackslash\hspace{0pt}}m{#1}}
\newcolumntype{C}[1]{>{\centering\let\newline\\arraybackslash\hspace{0pt}}m{#1}}
\newcolumntype{R}[1]{>{\raggedleft\let\newline\\arraybackslash\hspace{0pt}}m{#1}}
\setbeamertemplate{theorems}[numbered]
% Themes
 \mode<presentation> {
\usetheme{Hannover}
 \usecolortheme{default}
 \setbeamercovered{transparent}
 }

\setbeamercovered{transparent}

\setbeamertemplate{itemize item}{$\triangleright$}
\setbeamertemplate{itemize subitem}{$\diamond$}
\setbeamertemplate{enumerate items}[default]
\setbeamerfont{frametitle}{size=\large}
\PassOptionsToPackage{height=1cm}{beamerouterthemesidebar}
\usepackage{blindtext}

% Title
\title[Panel]{Regression with Panel Data \footnote[frame]{This section is based on \cite{Stock2020}, Chapter 10.}
}
\date[]{\today}


\author[Hao]{Jasmine(Yu) Hao}
\institute[VSE]{Faculty of Business and Economics\\Hong Kong University}


\begin{document}

\begin{frame}
\maketitle
\end{frame}

\begin{frame}{Motivation}
\begin{itemize}
\item
  Multiple regression is a powerful tool for controlling for the effect
  of variables on which we have data. \\
  If data are not available for some of the variables, however, they
  cannot be included in the regression, and the OLS estimators of the
  regression coefficients could have \textbf{omitted variable bias}.
\item
  Panel regression is a method for controlling for some types of omitted
  variables without actually observing them.
\item
  By studying changes in the dependent variable over time, it is
  possible to eliminate the effect of omitted variables that differ
  across entities but are constant over time.
\end{itemize}
\end{frame}

\begin{frame}{Example}

The empirical application in this chapter concerns drunk driving:

\begin{itemize}
\item
  What are the effects of alcohol taxes and drunk driving laws on
  traffic fatalities?

  \begin{itemize}
  \item
    We address this question using data on \emph{traffic fatalities,
    alcohol taxes, drunk driving laws}, and related variables for the
    \emph{48 contiguous U.S. states} from 1982 to 1988(\emph{7 years}).
  \item
    Unobserved variables:

    \begin{itemize}
    \item
      prevailing cultural attitudes toward drinking and
      driving(\emph{vary by entity}).
    \item
      improvements in the safety of new cars(\emph{vary through time}).
    \end{itemize}
  \end{itemize}
\end{itemize}
\end{frame}

\hypertarget{panel-data}{%
\section{Panel Data}\label{panel-data}}
\begin{frame}[allowframebreaks]{Panel Data}
\begin{itemize}
\item
  observations on the same \(n\) entities at two or more time periods
  \(T\), as is illustrated in Table 1.3. If the data set contains
  observations on the variables \(X\) and \(Y\), then the data are
  denoted\\
  \( (X_{it}, Y_{it}), i = 1,\ldots, n, \text{ and } t = 1,\ldots, T, (10.1) \)

  \begin{itemize}
  \item
    \(i\), refers to the \textbf{entity} being observed
  \item
    \(t\), refers to the \textbf{date} at which it is observed.
  \end{itemize}
\item
  The state traffic fatality data studied in this chapter are panel
  data. Those data are for \(n = 48\) entities (states), where each
  entity is observed in \(T = 7\) time periods (each of the years
  \(1982,\ldots ,1988\)), for a total of \(7 * 48 = 336\) observations.

  \begin{itemize}
  \item
    Compare cross-sectional data, \(i\) denote the entity.
  \item
    Panel data, we need some additional notation to keep track of both
    the entity and the time period.
  \item
    Use two subscripts rather than one: The first, \(i\), refers to the
    entity, and the second, t, refers to the time period of the
    observation. Thus \(Y_{it}\) denotes the variable \(Y\) observed for
    the ith of \(n\) entities in the \(t\)-th of \(T\) periods.
  \end{itemize}
\end{itemize}
\end{frame}

\begin{frame}[allowframebreaks]{Balanced v.s. unbalanced panel}

\begin{itemize}
\item
  A \textbf{balanced panel} has all its observations; that is, the
  variables are observed for each entity and each time period.
\item
  A panel that has some missing data for at least one time period for at
  least one entity is called an \textbf{unbalanced panel}.
\item
  E.g. The traffic fatality data set has data for all 48 contiguous U.S.
  states for all seven years, so it is \textbf{balanced}.
\item
  If we did not have data on fatalities for some states in 1983), then
  the data set would be \textbf{unbalanced}.
\item
  The methods presented focus on \textbf{balanced panel};
\item
  all these methods can be used with an unbalanced panel, although
  precisely how to do so in practice depends on the regression software
  being used.
\end{itemize}
\end{frame}

\begin{frame}[allowframebreaks]{Background}

\begin{itemize}
\item
  There are approximately 40,000 highway traffic fatalities each year in
  the United States. Approximately one-fourth of fatal crashes involve a
  driver who was drinking, and this fraction rises during peak drinking
  periods.
\item
  One study (Levitt and Porter, 2001) estimates that

  \begin{itemize}
  \item
    as many as \(25\%\) of drivers on the road between 1 a.m. and 3 a.m.
    have been drinking
  \item
    that a driver who is legally drunk is at least 13 times as likely to
    cause a fatal crash as a driver who has not been drinking.
  \end{itemize}
\end{itemize}
\end{frame}

\begin{frame}{From Model to Results}

\begin{itemize}
\item
  How effective various government policies designed to discourage drunk
  driving actually are in reducing traffic deaths?
\item
  The panel data set contains variables related to traffic fatalities
  and alcohol, including

  \begin{itemize}
  \item
    the number of traffic fatalities in each state in each
    year(\textbf{fatality rate}, which is the number of annual traffic
    deaths per 10,000 people in the population in the state)
  \item
    the type of drunk driving laws in each state in each year,
  \item
    the tax on beer in each state(``real'' tax on a case of beer, which
    is the beer tax, put into 1988 dollars by adjusting for inflation.).
  \end{itemize}
\end{itemize}
\end{frame}

\begin{frame}[allowframebreaks]{OLS regression}
\begin{figure}
	\centering
	\includegraphics[width=0.65\linewidth]{../figure/panel_1.png}
\end{figure}

Figure 10.1a is a scatterplot of the data for 1982 on two of these
variables, the fatality rate and the real tax on a case of beer.

A point in this scatterplot represents the fatality rate in 1982 and the
real beer tax in 1982 for a given state.

The estimated regression line is

\[\widehat{FatalityRate} = \underset{(0.15)}{2.01} + \underset{(0.13)}{0.15} BeerTax \text{ (1982 data) }.\]

\begin{itemize}
\item
  The coefficient on the real beer tax is positive but not statistically
  significant at the 10\% level.
\item
  We reexamine this relationship for another year(1988):
\end{itemize}

\[\widehat{FatalityRate} = \underset{(0.11)}{1.86} + \underset{(0.13)}{0.44} BeerTax \text{ (1988 data) }.\]

\begin{itemize}
\item
  The coefficient real beer tax is statistically significant at the 1\%
  level (the t-statistic is 3.43).
\item
  Curiously, the estimated coefficients for the 1982 and the 1988 data
  are positive: Taken literally, higher real beer taxes are associated
  with more, not fewer, traffic fatalities.
\item
  \textbf{Should we conclude that an increase in the tax on beer leads
  to more traffic deaths?}

  \begin{itemize}
  \item
    Not necessarily, because these regressions could have substantial
    \emph{omitted variable bias}.
  \item
    Many factors affect the fatality rate, including the quality of the
    automobiles driven in the state, whether the state highways are in
    good repair, whether most driving is rural or urban, the density of
    cars on the road, and whether it is socially acceptable to drink and
    drive.
  \item
    Any of these factors may be correlated with alcohol taxes.
  \end{itemize}
\item
  One approach to these potential sources of omitted variable bias would
  be to collect data on all these variables and add them to the annual
  cross-sectional regressions.
\item
  Unfortunately, some of these variables, such as the cultural
  acceptance of drinking and driving, might be very hard or even
  impossible to measure.
\item
  If these factors remain constant over time in a given state, however,
  then another route is available. Because we have panel data, we can,
  in effect, hold these factors constant even though we cannot measure
  them. To do so, we use OLS regression with fixed effects.
\end{itemize}
\end{frame}

\section[Difference Regression]{Panel Data with Two Time Periods: ``Before and After'' Comparisons}
\begin{frame}[allowframebreaks]{Panel Data with Two Time Periods: ``Before and After'' Comparisons}
	
\begin{itemize}
\item
  When data for each state are obtained for \(T = 2\) time periods, it
  is possible to compare values of the dependent variable when \(t=2\)
  to values when \(t=1\).

  \begin{itemize}
  \item
    Example? Cultural factors.
  \end{itemize}
\item
  By focusing on changes in the dependent variable, this ``before and
  after'' or ``differences'' comparison, in effect, holds constant the
  unobserved factors that differ from one state to the next but do not
  change over time within the state.
\item
  Let \(Z_i\) be a \textbf{time-invariant} variable that determines the
  fatality rate in the \(i\)-th state.

  \begin{itemize}
  \item
    local cultural attitude toward drinking and driving, which changes
    slowly and thus could be considered to be constant between 1982 and
    1988.
  \end{itemize}
\end{itemize}

The population linear regression relating \(Z_i\) and the real beer tax
to the fatality rate is

\[FatalityRate_{it} = \beta_0 + \beta_1 BeerTax_{it} + \beta_2 Z_i + u_i\]

\begin{itemize}
\item
  where \(u_{it}\) is the error term, \(i = 1, \ldots , n,\) and
  \(t = 1,\ldots, T.\) Because
\item
  \(Z_i\) does not change over time, the influence of \(Z_i\) can be
  eliminated by analyzing the change in the fatality rate between the
  two periods.
\end{itemize}
\end{frame}

\begin{frame}[allowframebreaks]{Mathematical Derivation: Bofore and After}

\[\begin{split}
FatalityRate_{i,1982} = \beta_0 + \beta_1 BeerTax_{it} + \beta_2 Z_i + u_{i,1982} \\
FatalityRate_{i,1988} = \beta_0 + \beta_1 BeerTax_{it} + \beta_2 Z_i + u_{i,1988}.\end{split}\]

\begin{itemize}
\item
  Interpretation: cultural attitudes toward drinking and driving affect
  the level of drunk driving and thus the traffic fatality rate in a
  state.

  \begin{itemize}
  \item
    If not change between 1982 and 1988, then they did not produce any
    change in fatalities in the state.
  \item
    Rather, any changes in traffic fatalities over time must have arisen
    from other sources.
  \item
    changes in the tax on beer and changes in the error term (which
    captures changes in other factors that determine traffic deaths).
  \end{itemize}
\end{itemize}
\end{frame}

\begin{frame}[allowframebreaks]{Difference Regression}
\[\begin{split}
	FatalityRate_{i,1988} - FatalityRate_{i,1982}\\ = \beta_1 (BeerTax_{i,1988} - BeerTax_{i,1982}) + u_{i,1988} - u_{i,1982}.\end{split}
\]

\begin{itemize}
\item
  The specification eliminates the effect of the unobserved variables
  \(Z_i\) that are constant over time.
\item
  In other words, analyzing changes in \(Y\) and \(X\) has the effect of
  controlling for variables that are constant over time, thereby
  eliminating this source of omitted variable bias.
\end{itemize}

\begin{figure}
\centering
\includegraphics[width=\linewidth]{../figure/panel_2.png}
\end{figure}

\begin{itemize}
\item
  Figure 10.2 presents a scatterplot of the change in the fatality rate
  between 1982 and 1988 against the change in the real beer tax between
  1982 and 1988 for the 48 states in our data set.
\item
  A point in Figure 10.2 represents the change in the fatality rate and
  the change in the real beer tax between 1982 and 1988 for a given
  state.
\item
  The estimated OLS regression line is\\
  \begin{align*} \widehat{FatalityRate_{i,1988} - FatalityRate_{i,1982}}
  = \\ - \underset{(0.065)}{0.072} - \underset{(0.36)}{1.04} (BeerTax_{i,1988} - BeerTax_{i,1982}) .  \end{align*}
\item
  Intercept: allows for the possibility that the mean change in the
  fatality rate, in the absence of a change in the real beer tax, is
  nonzero.

  \begin{itemize}
  \item
    For example, the negative intercept \((-0.072)\) could reflect
    improvements in auto safety between 1982 and 1988 that reduced the
    average fatality rate.
  \end{itemize}
\item
  In contrast to the cross-sectional regression results, the estimated
  effect of a change in the real beer tax is \textbf{negative}.
\item
  The hypothesis that the population slope coefficient is 0 is rejected
  at the 5\% significance level.
\item
  According to this estimated coefficient, an increase in the real beer
  tax by \$1 per case reduces the traffic fatality rate by 1.04 deaths
  per 10,000 people.
\item
  This estimated effect is very large:

  \begin{itemize}
  \item
    The average fatality rate is approximately 2 in these data,
  \item
    the estimate suggests that traffic fatalities can be cut in half by
    increasing the real tax on beer by \$1 per case.
  \end{itemize}
\item
  By examining changes in the fatality rate over time, the regression in
  controls for fixed factors such as cultural attitudes toward drinking
  and driving.
\end{itemize}
\end{frame}

\begin{frame}[allowframebreaks]{Issues}

\begin{itemize}
\item
  Time-varying factors that influence traffic safety, correlated with
  the real beer tax, then their omission will produce omitted variable
  bias.
\item
  for now cannot draw substantive conclusions about the effect of real
  beer taxes on traffic fatalities.
\item
  This ``before and after'' or ``differences'' analysis works when the
  data are observed in two different years.

  \begin{itemize}
  \item
    Not directly applicable when \(T=7\).
  \end{itemize}
\end{itemize}

\hypertarget{fixed-effects-regression}{%
\section{Fixed Effects Regression}\label{fixed-effects-regression}}

\begin{itemize}
\item
  Fixed effects regression is a method for controlling for omitted
  variables in panel data when the omitted variables vary across
  entities (states) but do not change over time.
\item
  Unlike the ``before and after'' comparisons of Section 10.2, fixed
  effects regression can be used when there are two or more time
  observations for each entity.
\item
  The fixed effects regression model has \(n\) different intercepts, one
  for each entity.

  \begin{itemize}
  \item
    represented by a set of binary (or indicator) variables.
  \item
    absorb the influences of all omitted variables that differ from one
    entity to the next but are constant over time.
  \end{itemize}
\end{itemize}
\end{frame}

\subsection{The Fixed Effects Regression Model}
\begin{frame}[allowframebreaks]{The Fixed Effects Regression Model}
\begin{itemize}
\item
  Consider the regression model in Equation (10.4) with the dependent
  variable(\(FatalityRate\)) and observed regressor (\(BeerTax\))
  denoted as \(Y_{it}\) and \(X_{it}\), respectively:
\end{itemize}

\[Y_{it} = \beta_0 + \beta_1 X_{it} + \beta_2 Z_{i} + u_{it},\]

\begin{itemize}
\item
  \(Z_i\) is an unobserved variable that varies from one state to the
  next but does not change over time

  \begin{itemize}
  \item
    for example, \(Z_i\) represents cultural attitudes toward drinking
    and driving.
  \end{itemize}
\item
  We want to estimate \(\beta_1\), the effect on \(Y\) of \(X\), holding
  constant the unobserved state characteristics \(Z\).
\end{itemize}

Because \(Z_i\) varies from one state to the next but is constant over
time, the population regression model can be interpreted as having n
intercepts, one for each state. Specifically, let
\(\alpha_i = \beta_0 + \beta_2 Z_i\):

\[Y_{it} = \beta_1 X_{it} + \alpha_i + u_{it}\]

\begin{itemize}
\item
  The regression is the fixed effects regression model, in which
  \(\alpha_1,\ldots,\alpha_n\), are treated as unknown intercepts to be
  estimated, one for each state.
\item
  The interpretation of \(\alpha_i\) as a state-specific intercept comes
  from considering the population regression line for the ith state;
  this population regression line is \(a_i + \beta_1 X_{it}\).
\item
  The slope coefficient of the population regression line, \(\beta_1\),
  is the same for all states, but the intercept of the population
  regression line varies from one state to the next.
\item
  Because the intercept \(\alpha_i\) in can be thought of as the
  ``effect'' of being in entity \(i\) (in the current application,
  entities are states), the terms \(\alpha_1,\ldots,\alpha_n\), an are
  known as entity fixed effects.
\item
  The variation in the entity fixed effects comes from omitted variables
  that, like \(Z_i\), that vary across entities but not over time.
\item
  The state-specific intercepts in the fixed effects regression model
  also can be expressed using binary variables to denote the individual
  states.
\item
  That population regression line was expressed mathematically using a
  single binary variable indicating one of the groups.

  \begin{itemize}
  \item
    If we had only two states in our data set, that binary variable
    regression model would apply here.
  \item
    Because we have more than two states, however, we need additional
    binary variables to capture all the state-specific intercepts.
  \end{itemize}
\item
  To develop the fixed effects regression model using binary variables,
  let \(D_{1i}\) be a binary variable that equals 1 when \(i = 1\) and
  equals 0 otherwise, let \(D_{2i}\) equal 1 when \(i = 2\) and equal 0
  otherwise, and so on.
\item
  Note: we cannot include all \(n\) binary variables plus a common
  intercept, the regressors will be perfectly multicollinear.

  \begin{itemize}
  \item
    we arbitrarily omit the binary variable \(D_{1i}\) for the first
    entity.
  \end{itemize}
\end{itemize}
\end{frame}

\begin{frame}[allowframebreaks]{Fixed Effect Regression}

\[Y_{it} = \beta_0 + \beta_1 X_{it} + \gamma_2 D_{2i} + \gamma_3 D_{3i}  + \ldots + + \gamma_n D_{ni}  + u_{it}, (10.11)\]

\begin{itemize}
\item
  where \(\beta_0, \beta_1, \gamma_2, \ldots, \gamma_n\) are unknown
  coefficients to be estimated.
\item
  the population regression equation for the first state is
  \(\beta_0 + \beta_1 X_{it}\), so \(\alpha_1 = \beta_0\). For the
  second and remaining states, it is \(\beta_0 + \beta_1 X_{it}\), so
  \(\alpha_i = \beta_0 + \gamma_i\) for \(i = 2,\ldots,n\).

  \begin{itemize}
  \item
    Thus there are two equivalent ways to write the fixed effects
    regression model.

    \begin{itemize}
    \item
      write in terms of \(n\) statespecific intercepts.
    \item
      a common intercept and \(n - 1\) binary regressors.
    \item
      In both formulations, the slope coefficient on \(X\) is the same
      from one state to the next.
    \end{itemize}
  \end{itemize}
\end{itemize}
\end{frame}

\subsubsection{Extension to multiple \(X\)'s}
\begin{frame}[allowframebreaks]{Extension to multiple \(X\)'s}
	
If there are other observed determinants of \(Y\) that are correlated
with \(X\) and that change over time, then these should also be included
in the regression to avoid omitted variable bias.

\hypertarget{estimation-and-inference}{%
\subsection{Estimation and Inference}\label{estimation-and-inference}}

\begin{itemize}
\item
  The binary variable specification of the fixed effects regression
  model can be estimated by OLS.
\item
  This regression has \(k + n\) regressors (the \(k\) X's, the \(n - 1\)
  binary variables, and the intercept), so in practice this OLS
  regression is tedious or, in some software packages, impossible to
  implement if the number of entities is large.
\item
  Econometric software therefore has special routines for OLS estimation
  of fixed effects regression models.
\item
  These special routines are equivalent to using OLS but faster because
  they employ some mathematical simplifications that arise in the
  algebra of fixed effects regression.
\end{itemize}

The ``entity-demeaned'' OLS algorithm. Regression software typically
computes the OLS fixed effects estimator in two steps.

\begin{enumerate}
\def\labelenumi{\arabic{enumi}.}
\item
  the entity-specific average is subtracted from each variable.
\item
  the regression is estimated using ``entity-demeaned'' variables.
\end{enumerate}

\pagebreak
Consider the case of a single regressor fixed effects model, and take
the average of both sides; then \(Y_i = \beta_1 X_i + a_i + u_i\), where
\(\bar Y_{i} = \sum_{t=1}^T Y_{it}\), and \(X_i\) and \(u_i\) are
defined similarly.

Then
\(Y_{it} - \bar Y_i = \beta_1 (X_{it} - \bar X_i) + (u_{it} - \bar u_i)\).
\\
Let \(\tilde Y_{it} = Y_{it} - \bar Y_i\),
\(\tilde X_{it} = X_{it} - \bar X_i\) and
\(\tilde u_{it} = u_{it} - \bar u_i\) accordingly, \\
\( \tilde Y_{it} = \beta_1 \tilde X_{it} + \tilde u_{it}. (10.14)\)\\
Thus \(\beta_1\) can be estimated by the OLS regression of the
``entity-demeaned'' variables

\[\tilde Y_{it} = \beta_1 \tilde X_{it} + \tilde u_{it}\]

This estimator is identical to the OLS estimator of \(\beta_1\) obtained
by estimation of the fixed effects model using \(n - 1\) binary
variables \textbf{(Exercise 19.6)}.
\end{frame}


\subsubsection[Comparison]{The ``before and after'' (differences) regression versus the binary variables specification}
\begin{frame}[allowframebreaks]{The ``before and after'' (differences) regression versus the binary variables specification}
	
\begin{itemize}
\item
  When \(T = 2\) the OLS estimator of \(\beta_1\) from the binary
  variable specification and that from the ``before and after''
  specification are \textbf{identical} if the intercept is excluded from
  the ``before and after'' specification.
\item
  Thus, when \(T = 2\), there are three ways to estimate b1 by OLS: the
  ``before and after'' specification in Equation (10.7) (without an
  intercept), the binary variable specification in Equation (textbook
  10.11), and the entitydemeaned specification in Equation (10.14).
  These three methods are equivalent; that is, they produce identical
  OLS estimates of \(\beta_1\) \textbf{(Exercise 10.11)}.
\end{itemize}
\end{frame}

\subsection[Application]{Application to Traffic Deaths}
\begin{frame}[allowframebreaks]{Application to Traffic Deaths}
	
The OLS estimate of the fixed effects regression line relating the real
beer tax to the fatality rate, based on all 7 years of data (336
observations), is

\[FatalityRate = \underset{(0.29)}{-0.66} BeerTax + state~fixed~effects\]

\begin{itemize}
\item
  Conventionally, the estimated state fixed intercepts are not listed.
\item
  the estimated coefficient in the fixed effects regression is negative

  \begin{itemize}
  \item
    higher real beer taxes are associated with fewer traffic deaths,
  \item
    the regression is different from the difference regression.
  \item
    The difference regression uses only the data for 1982 and 1988
    (specifically, the difference between those two years), whereas the
    fixed effects regression uses the data for all 7 years.
  \item
    Because of the additional observations, the standard error is
    smaller in the fixed effect regression.
  \end{itemize}
\item
  Including state fixed effects in the fatality rate regression lets us
  avoid omitted variables bias arising from omitted factors, such as
  cultural attitudes toward drinking and driving, that vary across
  states but are constant over time within a state.
\item
  Other (time-varying) factors could lead to omitted variables bias.

  \begin{itemize}
  \item
    For example, over this period cars were getting safer, and occupants
    were increasingly wearing seat belts;
  \item
    if the real tax on beer rose, on average, during the mid-1980s, then
    \(BeerTax\) could be picking up the effect of overall automobile
    safety improvements.
  \end{itemize}
\end{itemize}
\end{frame}

\section{Regression with Time Fixed Effects}
\begin{frame}[allowframebreaks]{Regression with Time Fixed Effects}

\begin{itemize}
\item
  Time fixed effects can control for variables that are constant across
  entities but evolve over time.

  \begin{itemize}
  \item
    Safety improvements in new cars are introduced nationally, they
    serve to reduce traffic fatalities in all states(omitted variable
    that changes over time.)
  \end{itemize}
\item
  The population regression with explicit the effect of automobile
  safety(\(S_t\)):
\end{itemize}

\[Y_{it} = \beta_0 + \beta_1 X_{it} + \beta_2 Z_i + \beta_3 S_t + u_{it},\]

\begin{itemize}
\item
  where \(S_t\) is unobserved and where the single \(t\) subscript
  emphasizes that safety changes over time but is constant across
  states.
\item
  \(\beta_3 S_t\) represents variables that determine \(Y_{it}\), if
  \(S_t\) is correlated with \(X_{it}\), then omitting \(S_t\) from the
  regression leads to omitted variable bias.
\end{itemize}
\end{frame}

\subsection{Time Effects Only}
\begin{frame}[allowframebreaks]{Time Effects Only}

\begin{itemize}
\item
  Suppose that the variables \(Z_i\) are not present, so that the term
  \(\beta_2 Z_i\) can be dropped, although the term \(\beta_3 S_t\)
  remains. Our objective is to estimate \(\beta_1\), controlling for
  \(S_t\).
\item
  Although \(S_t\) is unobserved, its influence can be eliminated
  because it varies over time but not across states
\item
  Recall: in the entity fixed effects model, the presence of \(Z_i\)
  leads to the fixed effects regression model: state-specific intercept.
\item
  Similarly, because \(S_t\) varies over time but not over states, the
  presence of \(S_t\) leads to a regression model in which each time
  period has its own intercept.
\end{itemize}

The time fixed effects regression model with a single \(X\) regressor is

\[Y_{it} = \beta_1 X_{it} + \lambda_t + u_{it}. (10.17)\]

\begin{itemize}
\item
  This model has a different intercept, \(\lambda_t\), for each time
  period: the ``effect'' on \(Y\) of time period \(t\).
\item
  The terms \(\lambda_1,\ldots, \lambda_T\) are known as \textbf{time
  fixed effects}.
\item
  The variation in the time fixed effects comes from omitted variables
  that, like \(S_t\) in, vary over time but not across entities.
\end{itemize}

The time fixed effects regression model be represented using \(T - 1\)
binary indicators

\[Y_{it} = \beta_0 + \beta_1 X_{it} + \delta_2 B2_t + \ldots + \delta_T BT_t + u_{it},\]

\begin{itemize}
\item
  where \(\delta_2,\ldots, \delta_T\) are unknown coefficients and where
  \(B2_t = 1\) if \(t = 2\) and \(B2_t = 0\) otherwise, etc.
\item
  The first binary variable \(B1_t\) is omitted to prevent perfect
  multicollinearity.
\end{itemize}
\end{frame}

\section{Both Entity and Time Fixed Effects}
\begin{frame}[allowframebreaks]{Both Entity and Time Fixed Effects}
\begin{itemize}
\item
  If some omitted variables are constant over time but vary across
  states (such as cultural norms), while others are constant across
  states but vary over time (such as national safety standards), then it
  is appropriate to include both entity (state) and time effects.
\item
  The combined entity and time fixed effects regression model is
\end{itemize}

\[Y_{it} = \beta_1 X_{it} + a_i + \lambda_t + u_{it},\]

\begin{itemize}
\item
  where \(a_i\) is the entity fixed effect and \(\lambda_t\) is the time
  fixed effect. This model can equivalently be represented using
  \(n - 1\) entity binary indicators and \(T - 1\) time binary
  indicators, along with an intercept:
\end{itemize}

\[Y_{it} = \beta_0 + \beta_1 X_{it} + \gamma_2 D2_i + \ldots + \gamma_n D_ni + \delta_2 B2_t + \ldots + \delta_T BT_t + u_{it}. \]

\end{frame}

%* Estimation: OLS with the additional time and entity binary variables.
%<!-- * Alternatively, in a balanced panel the coefficients on the $X$’s can be computed by first deviating $Y$ and the $X$’s from their entity and time-period means and then by estimating the multiple regression equation of deviated $Y$ on the deviated $X$’s. This algorithm, which is commonly implemented in regression software, eliminates the need to construct the full set of binary indicators that appear. 
%* An equivalent approach is to deviate $Y$, the $X$’s, and the time indicators from their entity (but not time-period) means and to estimate $k + T$ coefficients by multiple regression of the deviated Y on the deviated $X$’s and the deviated time indicators.  -->
%* If $T = 2$, the entity and time fixed effects regression can be estimated using the “before and after” approach, including the intercept in the regression. 
%* Thus the “before and after” regression reported, in which the change in $FatalityRate$ from 1982 to 1988 is regressed on the change in $BeerTax$ from 1982 to 1988 including an intercept, provides the same estimate of the slope coefficient as the OLS regression of $FatalityRate$ on $BeerTax$, including entity and time fixed effects, estimated using data for the two years 1982 and 1988.


\subsubsection[Application]{Application to traffic deaths}
\begin{frame}[allowframebreaks]{Application to traffic deaths}
Adding time effects to the state fixed effects regression results in the OLS estimate of the regression line:

\begin{align*}
	FatalityRate & = \underset{(0.36)}{-0.64} BeerTax 
	\\ & + State~Fixed~ Effects +
	Time~ Fixed~Effects. (10.21)
\end{align*}

\begin{itemize}

\item This specification includes the beer tax, 47 state binary variables (state fixed effects), 6 single-year binary variables (time fixed effects), and an intercept, so this regression actually has $1 + 47 + 6 + 1 = 55$ right-hand variables. 
\item The coefficients on the time and state binary variables and the intercept are not reported because they are not of primary interest. 
\item Including time effects has little impact on the coefficient on the real beer tax. 
\item Although this coefficient is less precisely estimated when time effects are included, it is still significant at the 10\%, but not the 5\%, significance level $(t = -0.64 / 0.36 = -1.78)$.
\item This estimated relationship between the real beer tax and traffic fatalities is immune to omitted variable bias from variables that are constant either over time or across states. However, many important determinants of traffic deaths do not fall into this category, so this specification could still be subject to omitted variable bias.
\end{itemize}
\end{frame}

\section{Inference}
\subsection[Inference]{The sampling distribution, standard errors, and statistical inference}
\begin{frame}[allowframebreaks]{The sampling distribution, standard errors, and statistical inference}

\begin{itemize}
	\item In multiple regression with cross-sectional data, if the four least squares assumptions hold, then the sampling distribution of the $OLS$ estimator is normal in large samples. 
	\item Similarly, in multiple regression with panel data, if the fixed effects regression assumptions holds, 
	\begin{itemize}
		  \item then the sampling distribution of the fixed effects OLS estimator is normal in large samples, 
		\item  the variance of that distribution can be estimated from the data, 
		\item construct $t$-statistics and confidence intervals. Given the standard error, statistical inference—testing hypotheses (including joint hypotheses using $F$-statistics) and constructing confidence intervals—proceeds in exactly the same way as in multiple regression with cross-sectional data. 
		
	\end{itemize}
\end{itemize}
\end{frame}
\subsubsection{Assumptions}
\begin{frame}{Assumptions}
Model:
\[ 
Y_{it} = \beta_1 X_{it} + \alpha_i + u_{it},i=1,\ldots,n,t=1,\ldots, T.
\]
\begin{enumerate}
\def\labelenumi{\arabic{enumi}.}
\item
  \(u_{it}\) has conditional mean 0:
  \(E[u_{it}|X_{i1},X_{i2}, \ldots, X_{iT}]=0.\)
\item
  \((X_{i1},X_{i2},\ldots,X_{iT},u_{i1},\ldots,u_{iT})\) are i.i.d draws
  from a joint distribution.
\item
  Large outliers are unlikely: \((X_{it},u_{it})\) have finite fourth
  moment.
\item
  There is no perfect collinearity.
\end{enumerate}
\end{frame}

\begin{frame}[allowframebreaks]{Assumptions}
These assumptions extend the four least squares assumptions for causal
inference, stated for cross-sectional data to panel data.

\begin{itemize}
\item
  The first assumption is that the error term has conditional mean 0
  given all \(T\) values of \(X\) for that entity.
\item
  This assumption plays the same role as the first least squares
  assumption for cross-sectional data and implies that there is no
  omitted variable bias.
\item
  The requirement that the conditional mean of uit not depend on any of
  the values of \(X\) for that entity---past, present, or future---adds
  an important subtlety beyond the first least squares assumption for
  cross-sectional data.

  \begin{itemize}
  \item
    This assumption is violated if current \(u_{it}\) is correlated with
    past, present, or future values of \(X\).
  \end{itemize}
\item
  The second assumption is that the variables are i.i.d. across entities
  for \(i = 1,\ldots, n\).

  \begin{itemize}
  \item
    The second assumption for fixed effects regression holds if entities
    are selected by simple random sampling from the population.
  \end{itemize}
\item
  The third and fourth assumptions for fixed effects regression are
  analogous to the third and fourth least squares assumptions for
  cross-sectional data.
\end{itemize}
\end{frame}

\begin{frame}[allowframebreaks]{Auto-correlation}
\begin{itemize}
\item
  If \(X_{it}\) is correlated with \(X_{is}\) for different values of
  \({s}\) and t---that is, if \(X_{it}\) is correlated over time for a
  given entity---then \(X_{it}\) is said to be autocorrelated
  (correlated with itself, at different dates) or serially correlated.
\item
  Autocorrelation is a pervasive feature of time series data: What
  happens one year tends to be correlated with what happens the next
  year.
\item
  In the traffic fatality example, \(X_{it}\), the beer tax in state
  \(i\) in year \(t\), is autocorrelated: Most of the time the
  legislature does not change the beer tax, so if it is high one year
  relative to its mean value for state \(i\), it will tend to be high
  the next year, too.
\item
  Similarly, it is possible to think of reasons why uit would be
  autocorrelated. Recall that \(u_{it}\) consists of time-varying
  factors that are determinants of \(Y_{it}\) but are not included as
  regressors, and some of these omitted factors might be autocorrelated.
\item
  For example, a downturn in the local economy might produce layoffs and
  diminish commuting traffic, thus reducing traffic fatalities for 2 or
  more years.
\item
  Similarly, a major road improvement project might reduce traffic
  accidents not only in the year of completion but also in future years.
  Such omitted factors, which persist over multiple years, produce
  autocorrelated regression errors.
\item
  Not all omitted factors will produce autocorrelation in \(u_{it}\);
  for example, severe winter driving conditions plausibly affect
  fatalities, but if winter weather conditions for a given state are
  independently distributed from one year to the next, then this
  component of the error term would be serially uncorrelated. In
  general, though, as long as some omitted factors are autocorrelated,
  then \(u_{it}\) will be autocorrelated.
\end{itemize}
\end{frame}
\subsection[Standard Errors]{Standard Errors for Fixed Effects Regression}
\begin{frame}[allowframebreaks]{Standard Errors for Fixed Effects Regression}

\begin{itemize}
\item
  If the regression errors are autocorrelated, then the usual
  \textbf{heteroskedasticity-robust} standard error formula for
  cross-section regression is not valid. One way to see this is to draw
  an analogy to heteroskedasticity.
\item
  In a regression with cross-sectional data, if the errors are
  heteroskedastic, then the homoskedasticity-only standard errors are
  not valid because they were derived under the false assumption of
  homoskedasticity. Similarly, if the errors are autocorrelated, then
  the usual standard errors will not be valid because they were derived
  under the false assumption of no serial correlation.
\item
  Standard errors that are valid if \(u_{it}\) is potentially
  heteroskedastic and potentially correlated over time within an entity
  are referred to as heteroskedasticity-and autocorrelation-robust (HAR)
  standard errors. The standard errors used in this chapter are one type
  of HAR standard errors, clustered standard errors.
\item
  The term clustered arises because these standard errors allow the
  regression errors to have an arbitrary correlation within a cluster,
  or grouping, but assume that the regression errors are uncorrelated
  across clusters. In the context of panel data, each cluster consists
  of an entity.
\item
  Thus clustered standard errors allow for heteroskedasticity and for
  arbitrary autocorrelation within an entity but treat the errors as
  uncorrelated across entities. That is, clustered standard errors allow
  for heteroskedasticity and autocorrelation in a way that is consistent
  with the second fixed effects regression assumption.
\item
  Like heteroskedasticity-robust standard errors in regression with
  cross-sectional data, clustered standard errors are valid whether or
  not there is heteroskedasticity, autocorrelation, or both. If the
  number of entities \(n\) is large, inference using clustered standard
  errors can proceed using the usual large-sample normal critical values
  for t-statistics and \(F_q\), critical values for \(F\)-statistics
  testing \(q\) restrictions.
\item
  In practice, there can be a large difference between clustered
  standard errors and standard errors that do not allow for
  autocorrelation of uit. For example, the usual (cross-sectional data)
  heteroskedasticity-robust standard error for the Beer-Tax coefficient
  is \(0.25\), substantially smaller than the clustered standard error,
  \(0.36\), and the respective t-statistics testing \(\beta_1 = 0\) are
  \(-2.51\) and \(-1.78\). The reason we report the clustered standard
  error is that it allows for serial correlation of uit within an
  entity, whereas the usual heteroskedasticity-robust standard error
  does not.
\end{itemize}
\end{frame}

\section[Application]{Drunk Driving Laws and Traffic Deaths}
\begin{frame}[allowframebreaks]{Drunk Driving Laws and Traffic Deaths}
\begin{itemize}
\item
  Alcohol taxes are only one way to discourage drinking and driving.
  States differ in their punishments for drunk driving, and a state that
  cracks down on drunk driving could do so by toughening driving laws as
  well as raising taxes.
\item
  If so, omitting these laws could produce omitted variable bias in the
  OLS estimator of the effect of real beer taxes on traffic fatalities,
  even in regressions with state and time fixed effects. In addition,
  because vehicle use depends in part on whether drivers have jobs and
  because tax changes can reflect economic conditions (a state budget
  deficit can lead to tax hikes), omitting state economic conditions
  also could result in omitted variable bias. In this section, we
  therefore extend the preceding analysis of traffic fatalities to
  include other driving laws and economic conditions.
\item
  The results are summarized in Table 10.1. The format of the table is
  the same as that of the tables of regression results in Chapters 7
  through 9: Each column reports a different regression, and each row
  reports a coefficient estimate and standard error, a 95\% confidence
  interval for the coefficients on the policy variables of interest, a
  \(F\)-statistic and p-value, or other information about the
  regression.
\end{itemize}
\end{frame}

\begin{figure}
\centering
\includegraphics[width=0.75\linewidth]{../figure/W4_12.png}
\end{figure}

\begin{frame}[allowframebreaks]{Comments}
\begin{itemize}
\item
  Column (1) in Table 10.1 presents results for the OLS regression of
  the fatality rate on the real beer tax without state and time fixed
  effects. As in the cross-sectional regressions for 1982 and 1988
  {[}Equations (10.2) and (10.3){]}, the coefficient on the real beer
  tax is positive (0.36): According to this estimate, increasing beer
  taxes increases traffic fatalities!
\item
  However, the regression in column (2) {[}reported previously as
  Equation (10.15){]}, which includes state fixed effects, suggests that
  the positive coefficient in regression (1) is the result of omitted
  variable bias (the coefficient on the real beer tax is -0.66). The
  regression \(R^2\) jumps from \(0.091\) to \(0.889\) when fixed
  effects are included; evidently, the state fixed effects account for a
  large amount of the variation in the data.
\item
  Little changes when time effects are added, as reported in column (3)
  {[}reported previously as Equation (10.21){]}, except that the beer
  tax coefficient is now estimated less precisely. The results in
  columns (1) through (3) are consistent with the omitted fixed
  factors---historical and cultural factors, general road conditions,
  population density, attitudes toward drinking and driving, and so
  forth---being important determinants of the variation in traffic
  fatalities across states.
\item
  The next four regressions in Table 10.1 include additional potential
  determinants of fatality rates along with state and time effects. The
  base specification, reported in column (4), includes variables related
  to drunk driving laws plus variables that control for the amount of
  driving and overall state economic conditions.
\item
  The first legal variables are the minimum legal drinking age,
  represented by three binary variables for a minimum legal drinking age
  of 18, 19, and 20 (so the omitted group is a minimum legal drinking
  age of 21 or older). The other legal variable is the punishment
  associated with the first conviction for driving under the influence
  of alcohol, either mandatory jail time or mandatory community service
  (the omitted group is less severe punishment). The three measures of
  driving and economic conditions are average vehicle miles per driver,
  the unemployment rate, and the logarithm of real (1988 dollars)
  personal income per capita (using the logarithm of income permits the
  coefficient to be interpreted in terms of percentage changes of
  income; see Section 8.2). The final regression in Table 10.1 follows
  the ``before and after'' approach of Section 10.2 and uses only data
  from 1982 and 1988; thus regression (7) extends the regression in
  Equation (10.8) to include the additional regressors.
\item
  The regression in column (4) has four interesting results.
\end{itemize}
\end{frame}
\begin{enumerate}
\def\labelenumi{\arabic{enumi}.}
\item
  Including the additional variables reduces the estimated effect of the
  beer tax from -0.64 in column (3) to -0.45 in column (4). One way to
  evaluate the magnitude of this coefficient is to imagine a state with
  an average real beer tax doubling its tax; because the average real
  beer tax in these data is approximately \$0.50 per case (in 1988
  dollars), this entails increasing the tax by \$0.50 per case. The
  estimated effect of a \$0.50 increase in the beer tax is to decrease
  the expected fatality rate by 0.45 * 0.50 = 0.23 deaths per 10,000.
  This estimated effect is large: Because the average fatality rate is 2
  deaths per 10,000, a reduction of 0.23 corresponds to reducing traffic
  deaths by nearly one-eighth. This said, the estimate is quite
  imprecise: Because the standard error on this coefficient is 0.30, the
  95\% confidence interval for this effect is
  \(-0.45 * 0.50 \pm 1.96 * 0.30 * 0.50 = ( -0.52, 0.08)\). This wide
  95\% confidence interval includes 0, so the hypothesis that the beer
  tax has no effect cannot be rejected at the 5\% significance level.
\item
  The minimum legal drinking age is precisely estimated to have a small
  effect on traffic fatalities. According to the regression in column
  (4), the 95\% confidence interval for the increase in the fatality
  rate in a state with a minimum legal drinking age of 18, relative to
  age 21, is \((-0.11, 0.17)\). The joint hypothesis that the
  coefficients on the minimum legal drinking age variables are 0 cannot
  be rejected at the 10\% significance level: The F-statistic testing
  the joint hypothesis that the three coefficients are 0 is 0.35, with a
  p-value of 0.786.
\item
  The coefficient on the first offense punishment variable is also
  estimated to be small and is not significantly different from 0 at the
  10\% significance level.
\item
  The economic variables have considerable explanatory power for traffic
  fatalities. High unemployment rates are associated with fewer
  fatalities: An increase in the unemployment rate by 1 percentage point
  is estimated to reduce traffic fatalities by 0.063 deaths per 10,000.
  Similarly, high values of real per capita income are associated with
  high fatalities: The coefficient is 1.82, so a 1\% increase in real
  per capita income is associated with an increase in traffic fatalities
  of 0.0182 deaths per 10,000. According to these estimates, good
  economic conditions are associated with higher fatalities, perhaps
  because of increased traffic density when the unemployment rate is low
  or greater alcohol consumption when income is high. The two economic
  variables are jointly significant at the 0.1\% significance level (the
  \(F\)-statistic is 29.62).
\end{enumerate}

\begin{itemize}
\item
  Columns (5) through (7) of Table 10.1 report regressions that check
  the sensitivity of these conclusions to changes in the base
  specification. The regression in column (5) drops the variables that
  control for economic conditions. The result is an increase in the
  estimated effect of the real beer tax, which becomes significant at
  the 5\% level, but there is no appreciable change in the other
  coefficients. The sensitivity of the estimated beer tax coefficient to
  including the economic variables, combined with the statistical
  significance of the coefficients on those variables in column (4),
  indicates that the economic variables should remain in the base
  specification.
\item
  The regressionin column (6) shows that the results in column (4) are
  not sensitive to changing the functional form when the three drinking
  age indicator variables are replaced by the drinking age itself. When
  the coefficients are estimated using the changes of the variables from
  1982 to 1988 {[}column (7){]}, as in Section 10.2, the findings from
  column (4) are largely unchanged except that the coefficient on the
  beer tax is larger and is significant at the 1\% level. The strength
  of this analysis is that including state and time fixed effects
  mitigates the threat of omitted variable bias arising from unobserved
  variables that either do not change over time (like cultural attitudes
  toward drinking and driving) or do not vary across states (like safety
  innovations). As always, however, it is important to think about
  possible threats to validity. One potential source of omitted variable
  bias is that the measure of alcohol taxes used here, the real tax on
  beer, could move with other alcohol taxes, which suggests interpreting
  the results as pertaining more broadly than just to beer. A subtler
  possibility is that hikes in the real beer tax could be associated
  with public education campaigns. If so, changes in the real beer tax
  could pick up the effect of a broader campaign to reduce drunk
  driving.
\end{itemize}

Taken together, these results present a provocative picture of measures
to control drunk driving and traffic fatalities. According to these
estimates, neither stiff punishments nor increases in the minimum legal
drinking age have important effects on fatalities. In contrast, there is
evidence that increasing alcohol taxes, as measured by the real tax on
beer, does reduce traffic deaths, presumably through reduced alcohol
consumption. The imprecision of the estimated beer tax coefficient
means, however, that we should be cautious about drawing policy
conclusions from this analysis and that additional research is
warranted.

\begin{frame}[allowframebreaks,noframenumbering]
	\frametitle{References}
	\bibliographystyle{apalike}
	\bibliography{library}
\end{frame}

\end{document}
