\RequirePackage{currfile}
\tolerance=5000
\documentclass[10pt, xcolor=x11names,compress,usenames,dvipsnames]{beamer}

\usepackage[english]{babel}

\usepackage[framemethod=TikZ]{mdframed}

\usepackage{amssymb,amsmath,amsfonts,eurosym,geometry,ulem,graphicx,caption,color,setspace,comment,footmisc,caption,pdflscape,subfigure,array,hyperref,upgreek,bbm,xcolor,float,amsthm,amsmath,verbatim,setspace,ulem,textpos,changepage,url,multirow,tikz,color, colortbl,numprint,mathrsfs,cancel,wrapfig,booktabs,threeparttable,ebgaramond,natbib}

\usetikzlibrary{fit,shapes.geometric}

\newcounter{nodemarkers}
\newcommand\circletext[1]{%
    \tikz[overlay,remember picture]
        \node (marker-\arabic{nodemarkers}-a) at (0,1.5ex) {};%
    #1%
    \tikz[overlay,remember picture]
        \node (marker-\arabic{nodemarkers}-b) at (0,0){};%
    \tikz[overlay,remember picture,inner sep=2pt]
        \node[draw,rectangle,red ,fit=(marker-\arabic{nodemarkers}-a.center) (marker-\arabic{nodemarkers}-b.center)] {};%
    \stepcounter{nodemarkers}%
}


\setbeamertemplate{footline}[frame number]


\normalem

\newcommand{\tsout}[1]{\text{\sout{$#1$}}}
\definecolor{Gray}{gray}{0.9}
\newcolumntype{g}{>{\columncolor{Gray}}c}

\newtheorem{remark}{Remark}
\def\mb{\mathbf}
\def\iid{\mathrm{i.i.d.}}
\def\bs{\boldsymbol}
\def\tbf{\textbf}
\def\t{^{\top}}
\def\E{\mathbbm{E}}
\def\bSig{\bs{\Sigma}}

\newcommand{\mcitet}[1]{\mbox{\citet{#1}}}
\newcommand{\mcitep}[1]{\mbox{\citep{#1}}}
\newcommand{\ind}{\mathbbm{1}}

\DeclareMathOperator{\vect}{vec}
\DeclareMathOperator{\vecth}{vech}



\newcommand{\R}{\mathbbm{R}}
\newcommand{\prob}{\mathbbm{P}}
\newcommand{\var}{\text{var}}

\newtheorem{assumption}{Assumption}

\newtheorem{hyp}{Hypothesis}
\newtheorem{subhyp}{Hypothesis}[hyp]
\renewcommand{\thesubhyp}{\thehyp\alph{subhyp}}

\newcommand{\red}[1]{{\color{red} #1}}
\newcommand{\blue}[1]{{\color{blue} #1}}

\makeatletter
\newenvironment<>{proofs}[1][\proofname]{%
    \par
    \def\insertproofname{#1\@addpunct{.}}%
    \usebeamertemplate{proof begin}#2}
  {\usebeamertemplate{proof end}}
\makeatother


\newcolumntype{L}[1]{>{\raggedright\let\newline\\arraybackslash\hspace{0pt}}m{#1}}
\newcolumntype{C}[1]{>{\centering\let\newline\\arraybackslash\hspace{0pt}}m{#1}}
\newcolumntype{R}[1]{>{\raggedleft\let\newline\\arraybackslash\hspace{0pt}}m{#1}}
\setbeamertemplate{theorems}[numbered]
% Themes
 \mode<presentation> {
\usetheme{Hannover}
 \usecolortheme{default}
 \setbeamercovered{transparent}
 }

\setbeamercovered{transparent}

\setbeamertemplate{itemize item}{$\triangleright$}
\setbeamertemplate{itemize subitem}{$\diamond$}
\setbeamertemplate{enumerate items}[default]
\setbeamerfont{frametitle}{size=\large}
\PassOptionsToPackage{height=1cm}{beamerouterthemesidebar}
\usepackage{blindtext}

% Title
\title[Introduction]{Overview \footnote[frame]{This section is based on \cite{Stock2020}, Chapter 1.}
}
\date[]{\today}


\author[Hao]{Jasmine(Yu) Hao}
\institute[VSE]{Faculty of Business and Economics\\Hong Kong University}


\begin{document}

\begin{frame}
\maketitle
\end{frame}

\begin{frame}[allowframebreaks]{Introduction}
Ask a half dozen econometricians what econometrics is, and you could get
a half dozen different answers.

\begin{itemize}
\item
  the science of testing economic theories
\item
  the set of tools used for forecasting future values of economic
  variables, such as a firm's sales, the overall growth of the economy,
  or stock prices
\item
  process of fitting mathematical economic models to real-world data
\item
  it is the science and art of using historical data to make numerical,
  or quantitative, policy recommendations in government and business.
\end{itemize}

In fact, all these answers are right. 

\pagebreak
At a broad level, econometrics is the science and art of using economic
theory and statistical techniques to analyze economic data.

\begin{itemize}
\item
  Econometric methods are used in many branches of economics, including
  finance, labor economics, macroeconomics, microeconomics, marketing,
  and economic policy.
\item
  Econometric methods are also commonly used in other social sciences,
  including political science and sociology.
\item
  This course introduces you to the core set of methods used by
  econometricians.
\item
  We will use these methods to answer a variety of specific,
  quantitative questions from the worlds of business and government
  policy.
\end{itemize}
\end{frame}


\section[Economic Questions]{Economic Questions We Examine}
\begin{frame}[allowframebreaks]{Economic Questions}
  
Many decisions in economics, business, and government hinge on
understanding relationships among variables in the world around us.

These decisions require quantitative answers to quantitative questions.

This text examines several quantitative questions taken from current
issues in economics. Four of these questions concern education policy,
racial bias in mortgage lending, cigarette consumption, and
macroeconomic forecasting.

\begin{enumerate}
  \item \tbf{Question \#1: Does Reducing Class Size Improve Elementary School Education?} \\
  Debate for reform of the U.S. public education system:  reduce class sizes at elementary schools increase students grades? 

\begin{itemize}
\item
  Reducing class size \tbf{costs money}: It requires hiring more teachers and, if the school is already at capacity, building more classrooms.
\item
  A decision maker contemplating hiring more teachers must weigh these costs against the benefits.
\item
  To weigh costs and benefits, however, the decision maker must have a \textbf{precise quantitative understanding} of the likely benefits.
\item
  How beneficial is the effect on basic learning of
  smaller classes? Is it possible that smaller class size
  actually has no effect on basic learning?
\end{itemize}

Common sense: more learning occurs when there are fewer students. common sense cannot provide a quantitative answer to the question of what exactly is the effect on basic learning of reducing class size.

\begin{itemize}
\item
  Examine empirical evidence---that
  is, evidence based on data---relating class size to basic learning in elementary schools.
\item
  Dataset: 420 California school districts in 1999.  
 \item  
  Data suggests: Students in districts with small class sizes
  tend to perform better on standardized tests than students in districts with larger classes.
\end{itemize}

\textbf{Causal relationship?} While this fact is consistent with the idea that smaller classes produce better test scores, it might simply reflect many other advantages that students in districts with small classes have over their counterparts in districts with large classes. 
\begin{itemize}
  \item districts with small class sizes tend to have wealthier residents, could have more opportunities for learning outside the classroom.
  \item It could be \textbf{these extra learning opportunities} that lead to higher test scores, not smaller class sizes.
\end{itemize}

\item \tbf{Question \#2: Is There Racial Discrimination in the Market for Home Loans?} \\
\begin{itemize}
\item
  Most people buy their homes with a mortgage, a large loan secured by
  the value of the home.
\item
  By law, U.S. lending institutions \textbf{cannot take race into
  account} when deciding to grant or deny a request for a mortgage:

  \begin{itemize}
  \item
    Applicants who are identical in all ways except their race should be equally likely to have their mortgage applications approved.
  \end{itemize}
\item
  In theory, then, there should be no racial bias in mortgage lending.
\end{itemize}

\begin{itemize}
\item
  Researchers at the Federal  Reserve Bank of Boston found (using data from the early 1990s)

  \begin{itemize}
  \item
    that 28\% of black applicants are denied mortgages,
  \item
    while only 9\% of white applicants are denied.
  \end{itemize}
\item
  Do these data indicate that, in practice, there is \textbf{racial bias} in mortgage lending? If so, how large is it?
\item
  The fact that more black than white applicants are denied in the Boston Fed data does not by itself provide evidence of discrimination  by mortgage lenders because the black and white applicants
  \textbf{differ in many ways other than their race}.
\item
  Before concluding that there is bias in the mortgage market, these
  data must be examined more closely to see if there is a difference in the probability of being denied for \tbf{otherwise identical applicants} and,
\item
  if so, how big is this difference?
\item
  Method to quantify the effect of race on the chance of obtaining a
  mortgage, holding constant other applicant characteristics, notably
  their ability to repay the loan.
\end{itemize}

\item \tbf{Question \#3: Does Healthcare Spending Improve Health Outcomes?}
\begin{itemize}
\item
  Avoidable deaths can be reduced and survival can be extended through
  the provision of healthcare.
\item
  Healthcare has other beneficial effects too, like the improvement of
  the health-related quality of life of individuals.
\item
  To these ends and more, a vast quantity of resources is devoted to the
  provision of healthcare worldwide.
\item
  Which part of healthcare do we spend most money?

  \begin{itemize}
  \item
    across countries both in absolute and per capita terms, as well as
    variations in health outcomes across countries, for example measured
    by life expectancy at birth.
  \end{itemize}
\item
  Putting aside concerns about iatrogenesis (the idea that healthcare is
  bad for your health), basic economics says that \textbf{more
  expenditure on healthcare should generally reduce avoidable
  mortality}.
\item
  But by how much? If the amount spent on healthcare increases by 1\%,
  by what percentage will avoidable mortality decrease? The percentage
  change in avoidable mortality resulting from a 1\% increase in
  healthcare expenditure is the spending elasticity for mortality.
\item
  If we want to reduce avoidable mortality, say, 20\% by increasing
  healthcare expenditure, then we need to know the spending elasticity
  for mortality to calculate the healthcare expenditure increase
  necessary to achieve this reduction in avoidable mortality.
\item
  A number of policy objectives are based on meeting targets based on
  avoidable mortality; 
  \begin{itemize}
    \item for example, one of the United Nations Development Programme's sustainable development goals is that all countries should aim to reduce ``under-5 mortality to at least as low as 25 per 1,000 live births.''
  \end{itemize}
\item
  But how should the goal be met: from expanding healthcare services or
  other services?
\item And if increasing healthcare spending is to form part of the mix of policies, by how much will it need to increase?
\end{itemize}

The answers to these can be obtained with estimates of the spending
elasticity for mortality.

To estimate the value, we must examine empirical evidence about the
returns to healthcare spending---either based on variations in spending
across countries or within countries over time. 

Two of the biggest challenges concern the \tbf{extensive heterogeneity} across countries.
\begin{itemize}
\item
   The first challenge is \tbf{observable heterogeneity}, which concerns factors that affect countries' mortality rates that may also be associated with healthcare expenditure,
   \begin{itemize}
    \item
    for example, the income per capita of each country. This can be
    controlled for using \tbf{multiple regression analysis}.
   \end{itemize}
\item
  The second and more troublesome challenge is the presence of
  \textbf{unobservable heterogeneity}.
  \begin{itemize}
    \item 
    Unobserved factors may be important in the underlying processes
    determining both how decisions are made on how much money is spent on
    healthcare, and how the overall level of health outcome that is
    attained.  
  \end{itemize}
\item
  \tbf{Simultaneously causality}: factors result in causality running in both
  directions---healthcare reduces mortality, but higher healthcare
  expenditure might be a response to unobserved factors.
\end{itemize}
\end{enumerate}

\end{frame}


\subsection[Quantitative]{Quantitative Questions, Quantitative Answers}
\begin{frame}[allowframebreaks]{Quantitative Questions, Quantitative Answers}
Each of these questions requires a \tbf{numerical answer}. Economic
theory provides clues about that answer---for example, cigarette
consumption ought to go down when the price goes up---but the actual
value of the number must be learned empirically, that is, by analyzing
data. 

Uncertainties in the answers:

\begin{itemize}
\item
  A different set of data would produce a different numerical answer.
\item
  Conceptual framework to measure of how precise the answer is.
\item
  multiple regression: quantify how a change in one variable affects another variable, holding other things constant.
\pagebreak
\item
  For example, what effect does a change in class size have on test
  scores, holding constant or controlling for student characteristics
  (such as family income) that a school district administrator cannot
  control?
\item
  What effect does your race have on your chances of having a mortgage
  application granted, holding constant other factors such as your
  ability to repay the loan? What effect does a 1\% increase in the
  price of cigarettes have on cigarette consumption, holding constant
  the income of smokers and potential smokers?
\end{itemize}
\end{frame}


\section{Causal Effects and Idealized Experiments}
\begin{frame}[allowframebreaks]{Causal Effects and Idealized Experiments}
  
\begin{itemize}
\item
  The questions concern causal relationships among variables. In common
  usage,

  \begin{itemize}
  \item
    an action is said to cause an outcome if the outcome is the direct
    result, or consequence, of that action.
  \item
    eg. Touching a hot stove causes you to get burned, drinking water causes
    you to be less thirsty, putting air in your tires causes them to
    inflate, putting fertilizer on your tomato plants causes them to
    produce more tomatoes.
  \end{itemize}
\item
  Causality means that a specific action (applying fertilizer) leads to
  a specific, measurable consequence (more tomatoes).
\end{itemize}
\end{frame}


\subsection{Estimation of Causal Effects}
\begin{frame}[allowframebreaks]{Estimation of Causal Effects}
  \begin{itemize}
\item
  How best might we measure the causal effect on tomato yield (measured
  in kilograms) of applying a certain amount of fertilizer, say, 100
  grams of fertilizer per square meter?
\item
  One way to measure this causal effect is to conduct an experiment.
\item
  In that experiment, a horticultural researcher plants many plots of
  tomatoes.
\item
  Each plot is tended identically, with one exception: Some plots get
  100 grams of fertilizer per square meter, while the rest get none.
\item
  Whether or not a plot is fertilized is determined randomly by a
  computer, ensuring that any other differences between the plots are
  unrelated to whether they receive fertilizer.
\item
  At the end of the growing season, the horticulturalist weighs the
  harvest from each plot.
\item
  The difference between the average yield per square meter of the
  treated and untreated plots is the effect on tomato production of the
  fertilizer treatment.
\item
  This is an example of a \textbf{randomized controlled experiment}.

  \begin{itemize}
  \item
    a \textbf{control group} that receives no treatment (no fertilizer)
  \item
    a \textbf{treatment group} that receives the treatment (100
    \(g/m^2\) of fertilizer). It is \textbf{randomized} in the sense
    that the treatment is assigned randomly.
  \end{itemize}
\item
  This random assignment eliminates the possibility of a systematic
  relationship between, for example, how sunny the plot is and whether
  it receives fertilizer so that the only systematic difference between
  the treatment and control groups is the treatment.
\item
  If this experiment is properly implemented on a large enough scale,
  then it will yield an estimate of the \textbf{causal effect} on the
  outcome of interest (tomato production) of the treatment (applying 100
  \(g/m^2\) of fertilizer).
\end{itemize}

The \textbf{causal effect} is defined to be the effect on an outcome of
a given action or treatment, as measured in an ideal randomized
controlled experiment. In such an experiment, the only systematic reason
for differences in outcomes between the treatment and control groups is
the treatment itself.



Experiments are used increasingly widely in econometrics. In many
applications, however, they are not an option because they are
unethical, impossible to execute satisfactorily, too time-consuming, or
prohibitively expensive. Even with nonexperimental data, the concept of
an ideal randomized controlled experiment is important because it
provides a definition of a causal effect.
\end{frame}



\section{Data: Sources and Types}


\subsection{Experimental versus Observational Data}
\begin{frame}[allowframebreaks]{Experimental versus Observational Data}
	
Experimental data come from experiments designed to evaluate a treatment
or policy or to investigate a causal effect. \\
For example, the state of Tennessee financed a large randomized
controlled experiment examining class size in the 1980s.

\begin{itemize}
\item
  In that experiment, which we examine in Chapter 13, thousands of
  students were randomly assigned to classes of different sizes for
  several years and were given standardized tests annually.
\item
  The Tennessee class size experiment cost millions of dollars and
  required the ongoing cooperation of many administrators, parents, and
  teachers over several years.
\end{itemize}

\begin{itemize}
\item
  Because real-world experiments with human subjects are difficult to
  administer and to control, they have flaws relative to ideal
  randomized controlled experiments. Moreover, in some circumstances,
  experiments are not only expensive and difficult to administer but
  also unethical. (Would it be ethical to offer randomly selected
  teenagers inexpensive cigarettes to see how many they buy?)
\item
  Because of these financial, practical, and ethical problems,
  experiments in economics are relatively rare. Instead, most economic
  data are obtained by observing real-world behavior.
\item
  Data obtained by observing actual behavior outside an experimental
  setting are called observational data. Observational data are
  collected using surveys, such as telephone surveys of consumers, and
  administrative records, such as historical records on mortgage
  applications maintained by lending institutions.
\end{itemize}

Observational data pose major challenges to econometric attempts to
estimate causal effects, and the tools of econometrics are designed to
tackle these challenges.

\begin{itemize}
\item
  In the real world, levels of ``treatment'' (the amount of fertilizer
  in the tomato example, the student--teacher ratio in the class size
  example) are not assigned at random, so it is difficult to sort out
  the effect of the ``treatment'' from other relevant factors.
\item
  Much of econometrics, and much of this text, is devoted to methods for
  meeting the challenges encountered when real-world data are used to
  estimate causal effects.
\item
  Whether the data are experimental or observational, data sets come in
  three main types: cross-sectional data, time series data, and panel
  data. In this text, you will encounter all three types.
\end{itemize}
\end{frame}

\hypertarget{cross-sectional-data}{%
\subsection{Cross-Sectional Data}\label{cross-sectional-data}}
\begin{frame}[allowframebreaks]{Cross-Sectional Data}
Data on different entities---workers, consumers, firms, governmental
units, and so forth--- for a single time period are called
cross-sectional data. For example, the data on test scores in California
school districts are cross sectional. Those data are for 420 entities
(school districts) for a single time period (1999). \\
In general, the number of entities on which we have observations is
denoted n; so, for example, in the California data set, n = 420.

\begin{itemize}
\item
  The California test score data set contains measurements of several
  different variables for each district. Some of these data are
  tabulated in Table 1.1.
\item
  Each row lists data for a different district. For example, the average
  test score for the first district (``district 1'') is 690.8; this is
  the average of the math and science test scores for all fifth-graders
  in that district in 1999 on a standardized test (the Stanford
  Achievement Test).
\item
  The average student--teacher ratio in that district is 17.89; that is,
  the number of students in district 1 divided by the number of
  classroom teachers in district 1 is 17.89.
\item
  Average expenditure per pupil in district 1 is \$6385. The percentage
  of students in that district still learning English---that is, the
  percentage of students for whom English is a second language and who
  are not yet proficient in English---is 0\%.
\item
  The remaining rows present data for other districts. The order of the
  rows is arbitrary, and the number of the district, which is called the
  \textbf{observation number}, is an arbitrarily assigned number that
  organizes the data. As you can see in the table, all the variables
  listed vary considerably.
\end{itemize}

\begin{figure}
\centering
\includegraphics[width=0.8\textwidth]{../figure/W0_1.png}
\end{figure}

Link to download the
data:\url{http://fmwww.bc.edu/ec-p/data/stockwatson/caschool.des}.
\end{frame}

\hypertarget{time-series-data}{%
\subsection{Time Series Data}\label{time-series-data}}
\begin{frame}[allowframebreaks]{Time Series Data}
	
Time series data are data for a single entity (person, firm, country)
collected at multiple time periods.

\begin{itemize}
\item
  Our data set on the growth rate of GDP and the term spread in the
  United States is an example of a time series data set.
\item
  The data set contains observations on two variables (the growth rate
  of GDP and the term spread) for a single entity (the United States)
  for 232 time periods.
\item
  Each time period in this data set is a quarter of a year (the first
  quarter is January, February, and March; the second quarter is April,
  May, and June; and so forth).
\item
  The observations in this data set begin in the first quarter of 1960,
  which is denoted 1960:Q1, and end in the fourth quarter of 2017
  (2017:Q4).
\item
  The number of observations (that is, time periods) in a time series
  data set is denoted \(T\). Because there are 232 quarters from 1960:Q1
  to 2017:Q4, this data set contains T = 232 observations.\\
 
\end{itemize}
\begin{figure}
  \centering
	 \includegraphics[width=0.8\textheight]{../figure/W0_2.png}
\end{figure}
\end{frame}


\subsection{Panel Data}

\begin{frame}[allowframebreaks]{Panel Data}

Panel data, also called \textbf{longitudinal data}, are data for
multiple entities in which each entity is observed at two or more time
periods. \\
Our data on cigarette consumption and prices are an example of a panel
data set, and selected variables and observations in that data set are
listed in Table 1.3.

\begin{figure}
\centering
\includegraphics[width=0.8\textheight]{../figure/W0_3.png}
\end{figure}

\begin{itemize}
\item
  The \textbf{number of entities} in a panel data set is denoted \(n\),
\item
  and the \textbf{number of time periods} is denoted \(T\).
\item
  In the cigarette data set, we have observations on \(n = 48\)
  continental U.S. states (entities) for \(T = 11\) years (time periods)
  from 1985 to 1995.
\item
  Thus, there is a total of \(n * T = 48 * 11 = 528\) observations.
\end{itemize}
\end{frame}

\begin{frame}[allowframebreaks,noframenumbering]
  \frametitle{References}
  \bibliographystyle{apalike}
\bibliography{library}
\end{frame}

\end{document}
